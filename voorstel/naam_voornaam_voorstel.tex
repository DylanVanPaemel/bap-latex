%==============================================================================
% Sjabloon onderzoeksvoorstel bachelorproef
%==============================================================================
% Gebaseerd op LaTeX-sjabloon ‘Stylish Article’ (zie voorstel.cls)
% Auteur: Jens Buysse, Bert Van Vreckem

\documentclass[fleqn,10pt]{voorstel}

%------------------------------------------------------------------------------
% Metadata over het voorstel
%------------------------------------------------------------------------------

\JournalInfo{HoGent Bedrijf en Organisatie}
\Archive{Bachelorproef 2018 - 2019} % Of: Onderzoekstechnieken

%---------- Titel & auteur ----------------------------------------------------

% TODO: geef werktitel van je eigen voorstel op
\PaperTitle{}
\PaperType{Onderzoeksvoorstel Bachelorproef} % Type document

% TODO: vul je eigen naam in als auteur, geef ook je emailadres mee!
\Authors{Steven Stevens\textsuperscript{1}} % Authors
\CoPromotor{Piet Pieters\textsuperscript{2} (Bedrijfsnaam)}
\affiliation{\textbf{Contact:}
  \textsuperscript{1} \href{mailto:steven.stevens.u1234@student.hogent.be}{steven.stevens.u1234@student.hogent.be};
  \textsuperscript{2} \href{mailto:piet.pieters@acme.be}{piet.pieters@acme.be};
}

%---------- Abstract ----------------------------------------------------------

\Abstract{Hier schrijf je de samenvatting van je voorstel, als een doorlopende tekst van één paragraaf. Wat hier zeker in moet vermeld worden: \textbf{Context} (Waarom is dit werk belangrijk?); \textbf{Nood} (Waarom moet dit onderzocht worden?); \textbf{Taak} (Wat ga je (ongeveer) doen?); \textbf{Object} (Wat staat in dit document geschreven?); \textbf{Resultaat} (Wat verwacht je van je onderzoek?); \textbf{Conclusie} (Wat verwacht je van van de conclusies?); \textbf{Perspectief} (Wat zegt de toekomst voor dit werk?).

Bij de sleutelwoorden geef je het onderzoeksdomein, samen met andere sleutelwoorden die je werk beschrijven.

Vergeet ook niet je co-promotor op te geven.
}

%---------- Onderzoeksdomein en sleutelwoorden --------------------------------
% TODO: Sleutelwoorden:
%
% Het eerste sleutelwoord beschrijft het onderzoeksdomein. Je kan kiezen uit
% deze lijst:
%
% - Mobiele applicatieontwikkeling
% - Webapplicatieontwikkeling
% - Applicatieontwikkeling (andere)
% - Systeembeheer
% - Netwerkbeheer
% - Mainframe
% - E-business
% - Databanken en big data
% - Machineleertechnieken en kunstmatige intelligentie
% - Andere (specifieer)
%
% De andere sleutelwoorden zijn vrij te kiezen

\Keywords{Onderzoeksdomein. Keyword1 --- Keyword2 --- Keyword3} % Keywords
\newcommand{\keywordname}{Sleutelwoorden} % Defines the keywords heading name

%---------- Titel, inhoud -----------------------------------------------------

\begin{document}

\flushbottom % Makes all text pages the same height
\maketitle % Print the title and abstract box
\tableofcontents % Print the contents section
\thispagestyle{empty} % Removes page numbering from the first page

%------------------------------------------------------------------------------
% Hoofdtekst
%------------------------------------------------------------------------------

% De hoofdtekst van het voorstel zit in een apart bestand, zodat het makkelijk
% kan opgenomen worden in de bijlagen van de bachelorproef zelf.
%---------- Inleiding ---------------------------------------------------------

\section{Introductie} % The \section*{} command stops section numbering
\label{sec:introductie}

Technologie wordt steeds belangrijker in ons leven.  We ge-bruiken technologie onder andere om ons leven aangenamerte maken.  Zowel in de Apple als de Play store zijn er veelapplicaties die mobile coaching aanbieden, het doel van dezeapplicaties is om een gezondere levensstijl te hanteren.  Uitonderzoek blijkt dat niet alle applicaties even kwaliteitsvolzijn.De bedoeling van dit onderzoek is om een voorstudie te doenvoor applicatie ontwikkelaars om betere mobile coaching ap-plicaties te bouwen met behulp van de community.Want veelgebruikers zijn teleurgesteld in deze applicaties en bekomenniet hun gewenst resultaat. Wat willen de gebruikers in dezeapplicaties?  Wat zijn absolute musts?  En kunnen we aande hand van de antwoorden voorspellen welke applicatie bijwelke gebruiker past met behulp van machine learning?

%---------- Stand van zaken ---------------------------------------------------

\section{State-of-the-art}
\label{sec:state-of-the-art}

In 2015 is een vergelijkende studie gedaan, namelijk \textcite{JMIR2015}. Het onderzoek bestudeerde mobile coaching applicaties die op dat moment op de markt waren. Er wordt gewerkt met een schaal waarin men de applicaties gaat vergelijken met vooraf vastgelegde richtlijnen. Uit de conclusie blijkt dat slechts een klein procent van de applicaties aan alle richtlijnen voldoet. Het is belangrijk bij mobile coaching applicaties dat de gebruiker centraal staat. \hfill  \break\break
Onder de noemer mobile coaching valt meer dan enkel bewegingsoefeningen, voor een gezonde levensstijl is veel meer nodig. In \textcite{EQUILIBRIO2005}  wordt ' life management system ' omschreven waarin krachtoefeningen en voedingsschema's centraal staan om een gezonde levensstijl te evenaren.\textcite{EQUILIBRIO2005} beschrijft dat je door deze combinatie een gelukkiger leven zal leiden. \hfill\break
 \break
We stellen dus vast dat veel mobiele applicaties niet voldoen aan fundamentele aanbevelingen voor de gezondheid. Er zijn 3 goede indicators waar ontwikkelaars voor dit type applicaties best rekening mee houden, namelijk: het FITT-principe\footnote{Frequency Intensity Type en Time’: hoe vaak , hoe intensief en hoe lang wordt er getraind?
}, veiligheid en structuur.

%---------- Methodologie ------------------------------------------------------
\section{Methodologie}
\label{sec:methodologie}

Dit onderzoek omtrent mobile coaching zal geen applicaties vergelijken maar gebruik maken van de community om te bepalen wat de gebruikers écht nodig hebben en wat niet. Daarom stellen we een vragenlijst op voor een mobile coaching applicatie: 
\begin{itemize}
\item Wat is uw geslacht?
\item Wat is uw leeftijd?
\item Hoe vaak sport u per week?
\item Hoe beoefent u meestal lichaamsbeweging?
\item Welk besturingssysteem heeft uw smartphone?
\item Hoe belangrijk vindt u het uitzicht en gebruiksgemak van een applicatie?
\item Heeft u al betalende applicaties gedownload?
\item Heeft u al een mobile coaching applicatie gebruikt?
\item Vindt u reclame in applicaties storend?
\item Ik kies voor een Mobile Coaching applicatie waarin ik...  
\begin{itemize}
\item enkel voedingsschema's krijg.
\item enkel trainingsschema's krijg.
\item voeding- en trainingschema's krijg.
\end{itemize}
\item Wat is uw hoofddoel bij het gebruik van een Mobile Coaching applicatie?
\item Ik vind extra uitleg of filmpjes bij oefeningen...?
\item Welke applicatie past best bij mij?\hfill \break \break 
\end{itemize}
 De vragenlijst zal hoofdzakelijk digitaal beschikbaar gesteld worden op \textcite{Facebook} en gezondheidsfora.  Voor het onderzoek zal een zelfgemaakte een PHP-platform voorzien worden waarin de antwoorden opgeslagen worden in een CSV-bestand. \hfill \break \break 
 In het tweede deel van het onderzoek zal een \textcite{Python} omgeving opgezet worden om machine learning toe te passen. Er zal een model met neuronen getraind worden met \textcite{Keras} en onderliggend \textcite{TensorFlow}. Het model zal dagelijks 2 keer getraind worden met data. Eens het model voldoende getraind is, zal het in staat zijn om voorspellingen te doen met nieuwe data.

%---------- Verwachte resultaten ----------------------------------------------
\section{Verwachte resultaten}
\label{sec:verwachte_resultaten}

Om correcte voorspellingen te doen met machine learning zijn veel resultaten noodzakelijk. Hoe meer antwoorden we krijgen uit onze vragenlijst, hoe accurater onze machine een voorspelling zal kunnen doen. Indien er niet genoeg resultaten zijn zal alles toch uitgewerkt worden maar dan als proof-of-concept.

%---------- Verwachte conclusies ----------------------------------------------
\section{Verwachte conclusies}
\label{sec:verwachte_conclusies}

De conclusie hangt af van het aantal deelnemers aan het onderzoek. Verder wordt er verwacht dat de gebruikers 'user experience' boven de functionaliteit verkiezen. Iets wat professioneel oogt zal meestal als goed in het algemeen bestempeld worden. Er wordt verwacht dat men duidelijk gebruikers kan onderverdelen in verschillende groepen aan de hand de vragenlijst. De vragenlijst kan verder los gebruikt worden als applicatie ontwikkelaar om rekening te houden met de vraag van de community.\hfill \break \break 


%------------------------------------------------------------------------------
% Referentielijst
%------------------------------------------------------------------------------
% TODO: de gerefereerde werken moeten in BibTeX-bestand ``voorstel.bib''
% voorkomen. Gebruik JabRef om je bibliografie bij te houden en vergeet niet
% om compatibiliteit met Biber/BibLaTeX aan te zetten (File > Switch to
% BibLaTeX mode)

\phantomsection
\printbibliography[heading=bibintoc]

\end{document}
