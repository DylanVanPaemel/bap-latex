%%=============================================================================
%% Voorwoord
%%=============================================================================

\chapter*{Woord vooraf}
\label{ch:voorwoord}

%% TODO:
%% Het voorwoord is het enige deel van de bachelorproef waar je vanuit je
%% eigen standpunt (``ik-vorm'') mag schrijven. Je kan hier bv. motiveren
%% waarom jij het onderwerp wil bespreken.
%% Vergeet ook niet te bedanken wie je geholpen/gesteund/... heeft

Ik koos voor het onderwerp ‘fitness’ omdat ik in het dagelijkse leven al meer dan 5 jaar bezig ben met deze sport. Doordat je gepassioneerd bent, leek het mij makkelijker om ook over dit onderwerp te schrijven in mijn bachelorproef. Op deze manier stond ik volledig achter mijn keuze zonder over iets te moeten schrijven waar ik weinig interesse in heb. 

Bijkomend is er gekozen om Machine Learning in het onderzoek te betrekken om op deze manier een leuk technisch aspect toe te voegen. Bovendien heb ik gekozen voor de afstudeerrichting e-business waar Machine Learning en Big Data zeker een rol spelen in de verkoopwereld. 

Dit onderzoek zou nooit tot stand kunnen komen zijn zonder de nodige begeleiding. Daarom bedank ik eerst en vooral mijn promotor Noemie Slaats voor de feedback bij het schrijven van deze bachelorproef. Daarnaast bedank ik ook graag mijn co-promotor Wouter Van Peteghem voor alle informatie en inspiratie. Ik bedank ook alle deelnemers aan de vragenlijst voor mijn onderzoek, zonder jullie was het niet gelukt om zoveel data te verkrijgen over dit onderwerp.

Ik wens u alvast veel leesplezier.

