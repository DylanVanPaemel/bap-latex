%%=============================================================================
%% Resultaten
%%=========================================================================
\chapter{Resultaten}
\label{ch:resultaten}

\section{Resultaten vragenlijst}
\label{sec:Resultaten vragenlijst}
In totaal zijn er 548 online deelnames geregistreerd voor de vragenlijst. In de onderstaande tabel worden de antwoorden overlopen. In de 2de kolom is de som van de antwoorden terug te vinden en de laatste kolom toont het procent van het antwoord op de vraag ten opzichte van het totaal. De analyse is gemaakt in Rstudio, het script is terug te vinden in de bijlages.
\begin{center}
\begin{tabular}{ |p{10cm}|p{2cm}|p{2cm}| }
 \hline
 \multicolumn{3}{|c|}{Algemene gegevens} \\
 \hline
 \textbf{Geslacht} & \textbf{N = 548} &\textbf{Procent}\\ 
 \hline
 Vrouwen   & 250    &45.6\%   \\
 Mannen &   298  &54.4\%   \\
 \hline
 \textbf{Leeftijd deelnemers in 2 klasses} & \textbf{N = 548} &\textbf{Procent}\\ 
 \hline
 leeftijd <= 25   & 300    &54.7\%   \\
 leeftijd > 25 &   248  &45.3\%   \\
 \hline
  \textbf{Hoe vaak sport u per week? } & \textbf{N = 548} &\textbf{Procent}\\ 
 \hline
 minder dan 1 keer   & 90    &16.4\%   \\
 1 keer &   66  &12\%   \\
 2 keer &   74  &13.6\%   \\
meer dan 2 keer &   318 &58\%   \\
 \hline
     \textbf{Heeft u al een mobile coaching applicatie gebruikt?} & \textbf{N = 548} &\textbf{Procent}\\ 
 \hline
Ja   & 292    &53.3\%   \\
Nee &   256  &46.7\%   \\
  \hline
\end{tabular}
\end{center}



\vspace{1cm}
\begin{center}
\begin{tabular}{ |p{10cm}|p{2cm}|p{2cm}| }
 \hline
    \textbf{Welk besturingssysteem heeft uw smartphone?} & \textbf{N = 548} &\textbf{Procent}\\ 
 \hline
IOS   & 206    &37.6\%   \\
Android &   330  &60.2\%   \\
overig &   12  &2.2\%   \\
 \hline
 \multicolumn{3}{|c|}{Functionele requirements} \\
 \hline
     \textbf{Vindt u het belangrijk om de applicatie met een smartwatch of dergelijke te connecteren?} & \textbf{N = 548} &\textbf{Procent}\\ 
 \hline
Ja   & 306    &55.8\%   \\
Nee &   242  &44.2\%   \\
 \hline
     \textbf{Is het belangrijk dat u de applicatie kan koppelen aan social media? (Facebook,Twitter,...)} & \textbf{N = 548} &\textbf{Procent}\\ 
 \hline
Ja   & 110    &20\%   \\
Nee &   438  &80\%   \\
 \hline
      \textbf{Vindt u het belangrijk dat een applicatie gebruik maakt van uw geolocatie (GPS)? bv. om uw traject te zien} & \textbf{N = 548} &\textbf{Procent}\\ 
 \hline
Ja   & 362    &66\%   \\
Nee &   186  &34\%   \\
 \hline
       \textbf{Ik vind extra uitleg of filmpjes bij oefeningen...} & \textbf{N = 548} &\textbf{Procent}\\ 
 \hline
Absoluut nodig   & 82    &15\%   \\
Soms handig &   414  &75.5\%   \\
Overbodig &   52  &9.5\%   \\
  \hline
 \multicolumn{3}{|c|}{Niet-functionele Requirements} \\
 \hline
        \textbf{Bent u bereid te betalen voor een applicatie?} & \textbf{N = 548} &\textbf{Procent}\\ 
 \hline
Ja   & 150    &27.4\%   \\
Nee &   398  &72.6\%   \\
 \hline
         \textbf{Hoe belangrijk vindt u het uitzicht en gebruiksgemak van een applicatie?} & \textbf{N = 548} &\textbf{Procent}\\ 
 \hline
Belangrijk   & 386    &70.4\%   \\
Niet bijzonder & 60   &11\%   \\
Ik kijk enkel naar de functies binnen de applicatie &   102  &18.6\%   \\
 \hline
          \textbf{Wat is uw hoofddoel bij het gebruik van een Mobile Coaching applicatie?} & \textbf{N = 548} &\textbf{Procent}\\ 
 \hline
Spieropbouw  & 180    &32.8\%   \\
Afvallen & 130   &25.2\%   \\
Fit blijven &   230  &42\%   \\
  \hline
 \multicolumn{3}{|c|}{Gebruikerstype} \\
 \hline
         \textbf{Welk type applicatie past best bij u?} & \textbf{N = 548} &\textbf{Procent}\\ 
 \hline
 Enkel voedingsschema & 134    &24.5\%   \\
Enkel trainingsschema & 216  &39.4\%   \\
Mobile Coach &   198  &36.1\%   \\
  \hline
\end{tabular}
  \caption{Tabel 4.1: Resultaten online vragenlijst}
\end{center}
\newpage
Uit de vragenlijst blijkt dat er iets meer mannen dan vrouwen hebben deelgenomen. 318 deelnemers gaan meer dan 2 keer per week sporten. Ongeveer de helft heeft ooit al een mobile coaching applicatie gebruikt. Er kan ook afgeleid worden dat Android het meest gebruikte besturingssysteem is met 60,2\% in dit onderzoek. Gevolgd door IOS met 37,6\%. 
Volgende requirements vallen op uit cijfers van dit onderzoek. Algemeen kan er gezegd worden dat maar liefst 80\% geen belang hecht om de coaching applicatie te koppelen aan sociale media. Ongeveer 72\% van de deelnemers ziet het ook niet zitten om te betalen voor een mobile coaching applicatie. 70\% van de deelnemers vindt het uitzicht en gebruiksgemak van de applicatie belangrijk. 362 van de 548 deelnemers vindt het gebruik van geolocatie belangrijk. Het hoofddoel bij het gebruik van een mobile coaching applicatie is voor 42\% om fit te blijven, gevolgd door spieropbouw met 32,8\% en als laatste afvallen met 25,2\%.

Als de deelnemers moeten kiezen tussen de 3 gebruikerstypes dan wordt het vaakst voor “enkel trainingsschema” gekozen gevolgd door “mobile coaching” en als laatste  “enkel voedingsschema’s”. 


\subsection{Algemene gegevens  van mannen en vrouwen uit de bevraging}
\label{sec:Resultaten vragenlijst}
In dit onderdeel wordt er een algemeen beeld gevormd over de bevraagde groep. Er wordt meer inzicht gegeven over de antwoorden op de algemene vragen per geslacht. Dit kan vooral handig zijn als ontwikkelaar om een doelgroep uit te kiezen bij het ontwikkelen van een applicatie. In de tabel wordt het aandeel van de subgroep ten opzichte van het totaal aantal deelnemers geplaatst, in dit geval 548 personen. 
\begin{center}
\begin{tabular}{ |p{10cm}|p{2cm}|p{2cm}| }
 \hline
 \textbf{Leeftijd deelnemers in 2 klasses} & \textbf{mannen} &\textbf{vrouwen}\\ 
 \hline
 leeftijd <= 25   & 36.9\%    &17.9\%   \\
 leeftijd > 25 &   17.5\%  &27.7\%   \\
 \hline
  \textbf{Hoe vaak sport u per week? } & \textbf{mannen} &\textbf{vrouwen}\\ 
 \hline
 minder dan 1 keer   & 6.6\%   & 9.9\%  \\
 1 keer &   6.6\%  & 5.5\%   \\
 2 keer &   4.6\%& 8.8\%   \\
meer dan 2 keer &  36.5\% & 21.5\%  \\
 \hline
   \textbf{Welk besturingssysteem heeft uw smartphone?}  & \textbf{mannen} &\textbf{vrouwen}\\ 
 \hline
IOS   & 19.3\%   &18.2\%   \\
Android &   35\%  &25.2\%   \\
Overig &   0\%  & 2.3\%  \\
 \hline
    \textbf{Heeft u al een mobile coaching applicatie gebruikt?}  & \textbf{mannen} &\textbf{vrouwen}\\ 
 \hline
Ja   & 29.8\%  &23.6\%   \\
Nee &   24.4\%  &22.2\%   \\
  \hline
\end{tabular}
\caption{4.1.1: Algemene gegevens  van mannen en vrouwen uit de bevraging}
\end{center}
\newpage
De algemene gegevens zijn ongeveer evenwichtig verdeeld tussen mannen en vrouwen. Er hebben wel bijna 37\% mannen jonger dan 26 jaar deelgenomen aan het onderzoek tegenover ongeveer 18\% vrouwen van dezelfde leeftijdscategorie. Zowel het grootste deel van de mannen ( 36.5\%) en vrouwen (21.5\%) gaat meer dan 2 keer per week sporten. Dit wil zeggen dat er heel wat mensen met ervaring hebben deelgenomen aan de enquête, wat een verrijking voor de antwoorden over de requirements kan betekenen. 


\subsection{Verschil in requirements tussen mannen en vrouwen}
\label{sec:Resultaten vragenlijst}
In onderstaande tabel worden de antwoorden van vrouwen en mannen tegenover elkaar geplaatst. Er wordt onderzocht als voor elk geslacht dezelfde functionaliteiten belangrijk zijn of als er significante verschillen zijn. De antwoorden die pro of contra zijn worden opgenomen in de tabel. In het geval dat er meer dan 2 antwoorden mogelijk waren op de vraag, zijn enkel de voorstanders en de tegenstanders opgenomen. De neutrale antwoorden zijn hier even uitgefilterd omdat we opzoek zijn naar duidelijke verschillen. De resultaten zijn terug te vinden in procent.
\begin{center}
\begin{tabular}{ |p{2.5cm}|p{2.3cm}|p{2.3cm}||p{2.8cm}|p{2.9cm}| }
 \hline
     \textbf{Requirement} & \textbf{mannen pro} & \textbf{vrouwen pro} & \textbf{mannen contra} & \textbf{vrouwen contra } \\
 \hline
Connectie met smartwatch   & 24.1\%    &31.8\% & 30.3\%    &13.8\%   \\
 \hline
Sociale media koppelen  & 10.9\%    &9.2\% & 43.4\%    & 36.5\%   \\
 \hline
 Geolocatie  & 36.5\%    &29.6\% & 17.9\%    & 16.1\%   \\
 \hline
  Video bij oefening  & 9.5\%    &5.5\% & 7.3\%    & 2.2\%   \\
 \hline
   Betalende applicatie  & 17.9\%    &9.5\% & 36.5\%    &  36.1\%   \\
 \hline
   uitzicht en gebruiksgemak  & 41.6\%    &28.8\% & 8.4\%    &  10.2\%   \\
 \hline
\end{tabular}
  \caption{Tabel 4.1.2: Verschil in requirements tussen mannen en vrouwen}
\end{center}
De meningen om externe apparaten te verbinden zoals smartwatches liggen verdeeld. Over het algemeen vinden vrouwen dit een grotere noodzaak dan mannen. Vrouwen en mannen hebben dezelfde mening over het koppelen van sociale media, maar ongeveer 10\% bij beide groepen vindt dat dit moet. Zowel een grote meerderheid van mannen en vrouwen vindt dat geolocatie een functie moet zijn. Over video’s bij oefeningen en betalende applicaties zijn vrouwen en mannen het opnieuw met elkaar eens want video’s vinden ze over het algemeen soms handig maar niet noodzakelijk en ze zijn allebei voorstanderd van gratis applicaties. Het valt wel op dat mannen met bijna 18\% en vrouwen slechts met 9\% kiezen om te betalen voor een applicatie. Mannen vinden het uitzicht en gebruiksgemak van een applicatie met 41.6\% toch belangrijker dan vrouwen met 28.8\%. Maar slechts een kleine groep van allebei vindt het uitzicht en gebruiksgemak niet onbelangrijk.
\subsection{Verschil in requirements tussen leeftijd}
\label{sec:Resultaten vragenlijst}
In dit onderzoek zijn de deelnemers opgesplitst in 2 groepen. De ene groep loopt tot en met een leeftijd van 25 en worden gezien als jongvolwassenen en jeugd. De andere groep vanaf 26 jaar wordt gezien als volwassenen. De keuze om te splitsen op deze leeftijd komt omdat de mediaan van de leeftijd op 25 jaar ligt. Er wordt dus gesplitst in de helft van alle leeftijden. Er wordt gekeken als de leeftijd afhangt van het antwoord die gegeven wordt op de vraag. De resultaten zijn terug te vinden in onderstaande tabel in procent. Opnieuw zijn de neutrale antwoorden er uitgefilterd om significante verschillen te zien tussen pro en contra. 
\begin{center}
\begin{tabular}{ |p{2.5cm}|p{2.3cm}|p{2.3cm}||p{2.8cm}|p{2.9cm}| }
 \hline
     \textbf{Requirement} & \textbf{ =< 25 pro} & \textbf{> 26 pro} & \textbf{=< 25 contra} & \textbf{> 26 contra } \\
 \hline
Connectie met smartwatch   &24.4\%    &31.8\% & 30.2\%    &13.6\%   \\
 \hline
Sociale media koppelen  & 8\%    &12\% & 46.7\%    & 33.3\%   \\
 \hline
 Geolocatie  & 37.6\%    &28.4\% & 17.2\%    & 16.8\%   \\
 \hline
  Video bij oefening  & 9.9\%    &5.1\% & 6.9\%    & 2.6\%   \\
 \hline
   Betalende applicatie  & 14.2\%    &13.1\% &41.5\%    &  31.2\%   \\
 \hline
   uitzicht en gebruiksgemak  & 41.6\%    &28.8\% & 8.8\%    &  9.8\%   \\
 \hline
\end{tabular}
  \caption{Tabel 4.1.2: Verschil in requirements tussen leeftijd}
\end{center}
Opnieuw liggen de meningen over de connectie met externe apparaten verspreid. De volwassengeneratie vindt deze functionaliteit net iets belangrijker dan de jongvolwassenen. Het koppelen van sociale media is voor beide groepen geen belangrijke functie. De jongvolwassenen vinden voor 37.6\% dat geolocatie een must is en de volwassenen vinden dit voor 28.4\%. Er kan dus afgeleid worden dat hier ook beide groepen het met elkaar eens zijn. Zowel beide groepen zijn niet voor en niet tegen video’s bij oefeningen. Ze zien het eerder als een handige functie maar zeker niet als een verplichting. Beide groepen zijn ook grotendeel tegen betalende applicaties met 41.5\% voor de jongvolwassenen en 31.2\% voor de volwassenen. De jongvolwassenen hechten het meest belang aan het uitzicht en gebruiksgemak van een applicatie met 41.6\% en de volwassenen met 28.8\%. Opnieuw vinden beide groepen het uitzicht van een applicatie niet onbelangrijk want slechts 8.8\% van de jongvolwassen en 9.8\% van de volwassenen kijken enkel naar de functies binnen de applicatie zonder te letten op het uitzicht. 

\newpage
\section{Opstellen van de kritische punten}
\label{sec:Resultaten vragenlijst}