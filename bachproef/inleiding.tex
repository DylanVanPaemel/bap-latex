%%=============================================================================
%% Inleiding
%%=============================================================================

\chapter{Inleiding}
\label{ch:inleiding}

Het is normaal dat mensen verschillende interesses hebben, maar sommige zaken zoals een gezonde levensstijl zijn essentieel. Niet voldoende beweging is een vaak voorkomende oorzaak van aandoeningen. Bovendien willen mensen er lichamelijk graag goed uitzien, dit zit nu eenmaal in onze cultuur ingebakken. Om er goed uit te zien is sporten of een dieet volgen een logische keuze. De klik maken om te beginnen met een sport of te starten met een dieet is een stap in de goede richting, maar dit volhouden is een ander verhaal. Mensen gaan dus op zoek naar hulpmiddelen om hun extra goed te helpen en te motiveren bij hun traject. Ze zoeken een manier waarop ze dit kunnen blijven volhouden.

Een hulpmiddel kan in dit geval een mobiele applicatie zijn. Het gebruik van een applicatie is geen vreemde keuze omdat er meer dan 2,5 biljoen smartphone gebruikers op de wereld zijn. ~\autocite{Statista2018}

Als mensen de stap maken om gebruik te maken van een mobiele coaching applicatie dan staan ze nog voor een hele grote keuze want er bestaan namelijk 250 000 applicaties in de rubriek fitness \& gezondheid.  ~\autocite{Statista2018}

Als mensen de stap maken om gebruik te maken van een mobiele coaching applicatie dan staan ze nog voor een hele grote keuze want er bestaan namelijk 250 000 applicaties in de rubriek fitness \& gezondheid ~\autocite{Statista2018}. Gebruikers hebben dus een overvloed aan applicaties waaruit ze kunnen kiezen. In dit onderzoek zullen een aantal van de grootste applicaties gekozen worden en beoordeeld worden volgens kritische punten
\newpage
\section{Probleemstelling}
\label{sec:probleemstelling}

Zoals eerder vermeld worden mobile coaching applicaties gebruikt als hulpmiddel om extra te motiveren maar er zijn nog andere redenen om dergelijke applicaties te gebruiken.

Gebruikers kunnen al hun gegevens op een centrale plaats opslaan en raadplegen wat het gebruik van deze applicaties makkelijk en comfortabel maakt. Als gebruikers alles in de applicatie bijhouden dan kunnen ze hun progressie visueel zien. Bijvoorbeeld zien gebruikers dat ze minder lang doen om dezelfde afstand te lopen dan een maand geleden. Dit geeft een boost aan het zelfvertrouwen! Waardoor mensen zich niet alleen fysiek maar ook mentaal zullen beter voelen.

Mobile coaching applicaties laten ook toe om ‘goals’ in te stellen. Stel dat je aan het trainen bent voor een marathon of je wil wat kilo’s kwijt dan helpen deze applicaties de gebruikers. Het is soms moeilijk om zelf stappen te bepalen waarmee je jouw training moet verzwaren of aanpassen om progressie te boeken, daarom zijn dergelijke applicaties er om jou verder te helpen en stelselmatig jouw doel te bereiken.

Door de grote opkomst van smartphones is de mobiele industrie voor deze gezondheidsapplicaties onmisbaar geworden. Door deze digitale vooruitgang is het niet alleen makkelijk als eindgebruiker om onze gezondheid bij te houden maar bijvoorbeeld ook voor artsen die hun patiënten in de gaten willen houden op deze manier. Hierdoor staan artsen en hun patiënten sneller en makkelijker met elkaar in contact en dat levert vele voordelen op.

Mobile coaching biedt veel mogelijkheden om mensen te helpen met hun gezondheid, maar helaas zijn ook een aantal nadelen aan deze apps verbonden.  Enkele nadelen hier van zijn:

‘Iedere’ ontwikkelaar kan een mobile coaching applicatie publiceren in de Google Play store of Apple Store zonder te bewijzen dat hij/zij de kennis heeft over de gezondheid van de mens. Dus het is niet verstandig om zomaar een applicatie te volgen zonder grondig na te gaan wie deze applicatie heeft ontwikkeld.

Een 2de logisch nadeel is dus dat we de kwaliteit van de applicaties in vraag kunnen stellen, want werken en helpen alle applicaties in de Google Play Store of Apple Store mensen hun doel te bereiken?

\newpage

\section{Onderzoeksvraag}
\label{sec:onderzoeksvraag}

In het eerste deel zal er een beschrijvend onderzoek gebeuren. Hier proberen we een antwoord te vinden op de volgende hoofdvraag “wat houdt een goede mobile coaching applicatie in?” deze vraag zal beantwoord worden door het opstellen van kritische punten. Hieruit volgt de volgende deel vraag: Welke applicaties zijn op de markt en voldoen deze applicatie aan de kritische punten? 

Voor het tweede deel van het onderzoek wordt er een antwoord gezocht op de vraag: “ Is er een verband tussen het doel van de gebruiker en de keuze van een type applicatie? “

\section{Onderzoeksdoelstelling}
\label{sec:onderzoeksdoelstelling}

Het doel van dit onderzoek is om te kunnen bepalen aan welke functionaliteiten en requirements mobile coaching applicaties moeten voldoen om de gebruiker tevreden te stellen bij het gebruik van dergelijke applicaties. 

In dit onderzoek zullen er dus applicaties vergeleken worden met vooraf bepaalde kritische punten. Doordat er gebruik gemaakt wordt van een vragenlijst die ingevuld wordt door mensen met interesse tot het fitnessdomein, zullen de kritische punten grotendeels hierop gebaseerd zijn. De gekozen kritische punten dienen als indicatie om de kwaliteit van toekomstige applicaties te vergroten.  

Als er bestaande applicaties vergeleken worden wil het niet zeggen dat de applicatie die het best scoort volgens de kritische punten binnen dit onderzoek de beste applicatie is die op dit moment op de markt is. 

Doordat er een heel grote markt is voor dit type applicaties is de keuze voor de gebruiker niet makkelijk om de juiste applicatie te kiezen die bij jou past. Je wordt overspoeld door potentiële applicaties die je kan gebruiken om je te laten coachen.

Daarom zal er ook in dit onderzoek een verband gezocht worden tussen het doel van de gebruiker en het type applicatie waarvoor de gebruiker kiest. In dit onderzoek proberen we dit te voorspellen om de keuze van de gebruiker makkelijker te maken. 

De gebruiker zal enkele vragen moeten beantwoorden en op het einde krijgt de gebruiker een type applicatie + de beste applicaties binnen dit type (volgens de gekozen kritische punten) toegewezen die het best bij zijn eisen past.

\newpage
\section{Opzet van deze bachelorproef}
\label{sec:opzet-bachelorproef}

% Het is gebruikelijk aan het einde van de inleiding een overzicht te
% geven van de opbouw van de rest van de tekst. Deze sectie bevat al een aanzet
% die je kan aanvullen/aanpassen in functie van je eigen tekst.

Het vervolg van deze bachelorproef is als volgt opgebouwd:
In Hoofdstuk~\ref{ch:literatuurstudie}  wordt de literatuurstudie van mobile coaching beschreven.

In Hoofdstuk~\ref{ch:methodologie} wordt de methodologie besproken om een antwoord te kunnen geven op de onderzoeksvragen.

In Hoofdstuk~\ref{ch:resultaten} worden de resultaten van dit onderzoek besproken. De vragenlijst en mobile coaching applicaties zullen hier geanalyseerd worden.

In Hoofdstuk~\ref{ch:conclusie}, tenslotte, wordt een conclusie gevormd over de onderzoeksvragen van dit onderzoek.
