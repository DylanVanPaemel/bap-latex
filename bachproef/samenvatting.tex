%%=============================================================================
%% Samenvatting
%%=============================================================================

% TODO: De "abstract" of samenvatting is een kernachtige (~ 1 blz. voor een
% thesis) synthese van het document.
%
% Deze aspecten moeten zeker aan bod komen:
% - Context: waarom is dit werk belangrijk?
% - Nood: waarom moest dit onderzocht worden?
% - Taak: wat heb je precies gedaan?
% - Object: wat staat in dit document geschreven?
% - Resultaat: wat was het resultaat?
% - Conclusie: wat is/zijn de belangrijkste conclusie(s)?
% - Perspectief: blijven er nog vragen open die in de toekomst nog kunnen
%    onderzocht worden? Wat is een mogelijk vervolg voor jouw onderzoek?
%
% LET OP! Een samenvatting is GEEN voorwoord!

%%---------- Nederlandse samenvatting -----------------------------------------
%
% TODO: Als je je bachelorproef in het Engels schrijft, moet je eerst een
% Nederlandse samenvatting invoegen. Haal daarvoor onderstaande code uit
% commentaar.
% Wie zijn bachelorproef in het Nederlands schrijft, kan dit negeren, de inhoud
% wordt niet in het document ingevoegd.

\IfLanguageName{english}{%
\selectlanguage{dutch}
\chapter*{Samenvatting}
\selectlanguage{english}
}{}

%%---------- Samenvatting -----------------------------------------------------
% De samenvatting in de hoofdtaal van het document

\chapter*{\IfLanguageName{dutch}{Samenvatting}{Abstract}}

Tegenwoordig zijn er veel apps op de markt, ook voor mensen te helpen met coaching. Mensen zoeken een hulplijn om gemotiveerd te blijven bewegen en dan is een app kiezen die je steunt bij het trainen een ideale oplossing. Als je een applicatie wilt downloaden, wordt de gebruiker overspoeld met mogelijke apps. In dit onderzoek zijn een aantal applicaties eruit geplukt om een vergelijkende studie te maken. Daarvoor is er een vragenlijst online geplaatst zodat gebruikers de mogelijkheid hadden om te vertellen wat zij belangrijk en minder belangrijk vinden in dergelijke applicaties. Uit deze vragenlijst en ook de literatuurstudie zijn vervolgens kritische punten opgemaakt waar de applicaties op gecontroleerd zijn.

In het tweede deel van het onderzoek is Machine Learning toegepast op de data van de vragenlijst. Een aantal vragen werden gezien als inputdata en de laatste vraag bepaalde welk type gebruiker dit was. Op basis van het trainen van een neuraal netwerk is het mogelijk om zelf voorspellingen te doen over het gebruikerstype. 

De vergelijkende studie in dit onderzoek is bedoeld voor applicatie ontwikkelaars voor gezondheid apps die graag zouden weten op welke functionaliteiten ze zich moeten focussen om gebruikers voor zich te winnen. 

