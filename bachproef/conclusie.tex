%%=============================================================================
%% Conclusie
%%=============================================================================

\chapter{Conclusie}
\label{ch:conclusie}

%% TODO: Trek een duidelijke conclusie, in de vorm van een antwoord op de
%% onderzoeksvra(a)g(en). Wat was jouw bijdrage aan het onderzoeksdomein en
%% hoe biedt dit meerwaarde aan het vakgebied/doelgroep? Reflecteer kritisch
%% over het resultaat. Had je deze uitkomst verwacht? Zijn er zaken die nog
%% niet duidelijk zijn? Heeft het onderzoek geleid tot nieuwe vragen die
%% uitnodigen tot verder onderzoek?

Er kan besloten worden dat er duidelijk requirements zijn die gebruikers graag in een mobile coaching applicatie zien. De vragenlijst heeft ontwikkelaars meer inzicht gegeven waarop ze zich meer moeten focussen bij het ontwikkelen van dergelijke applicaties om gebruikers voor zich te winnen. 

Deze kritische punten die we in dit onderzoek bekomen zijn, zijn zeker niet de enige maar het kan zeker een goede basis zijn voor een verder onderzoek voor dit domein. Dit onderzoek heeft vooral de technische kant van een mobile coaching applicatie besproken. Dit is niet het enige wat belangrijk is, er zijn vele andere aspecten denk maar aan de gezondheid van de eindgebruiker. Om in dit een onderzoek een uitspraak te doen als de apps gezond zijn of niet, is er te weinig voorkennis van de onderzoeker over het medisch domein. Vandaar dat er wel een nieuwe onderzoeksvraag is gekomen. Zijn deze applicaties medisch verantwoord? 
De kritische punten die uit de vragenlijst en literatuurstudie zijn gehaald, zijn vervolgens toegepast op een paar gekozen applicaties van de 3 besproken categorieën. Er kan besloten worden dat er algemeen weinig achtergrondinformatie terug te vinden is van de ontwikkelaars en eventuele bewegingsspecialisten die de applicatie ontworpen hebben. Op deze manier zakt het vertrouwen en geloofwaardigheid van de applicatie. 

Dergelijke coaching applicaties bieden veel potentieel om mensen te laten bewegen, maar er is zeker nog ruimte voor verfijning. Zowel op gebied van begeleiding als de onderbouwde achtergrond  over  de makers en het team achter de applicatie.

In het tweede deel van het onderzoek is een machine gebouwd die na voldoende training voorspellingen kon maken op nooit geziene data. Dit was met een foutpercentage van 1\%. Op zich is dit nog te veel, maar het kon niet anders omdat er slecht beperkte data beschikbaar was. Als er meer data beschikbaar was om de machine te trainen dan waren de resultaten ook preciezer geweest. Daarom dat we het ontwerp als een ‘proof of concept’ kunnen beschouwen, omdat het echter wel mogelijk is om het gebruikerstype te bepalen mits er nog meer data beschikbaar zou zijn. 


